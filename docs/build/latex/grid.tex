%% Generated by Sphinx.
\def\sphinxdocclass{report}
\documentclass[letterpaper,10pt,french]{sphinxmanual}
\ifdefined\pdfpxdimen
   \let\sphinxpxdimen\pdfpxdimen\else\newdimen\sphinxpxdimen
\fi \sphinxpxdimen=.75bp\relax

\PassOptionsToPackage{warn}{textcomp}
\usepackage[utf8]{inputenc}
\ifdefined\DeclareUnicodeCharacter
% support both utf8 and utf8x syntaxes
  \ifdefined\DeclareUnicodeCharacterAsOptional
    \def\sphinxDUC#1{\DeclareUnicodeCharacter{"#1}}
  \else
    \let\sphinxDUC\DeclareUnicodeCharacter
  \fi
  \sphinxDUC{00A0}{\nobreakspace}
  \sphinxDUC{2500}{\sphinxunichar{2500}}
  \sphinxDUC{2502}{\sphinxunichar{2502}}
  \sphinxDUC{2514}{\sphinxunichar{2514}}
  \sphinxDUC{251C}{\sphinxunichar{251C}}
  \sphinxDUC{2572}{\textbackslash}
\fi
\usepackage{cmap}
\usepackage[T1]{fontenc}
\usepackage{amsmath,amssymb,amstext}
\usepackage{babel}



\usepackage{times}
\expandafter\ifx\csname T@LGR\endcsname\relax
\else
% LGR was declared as font encoding
  \substitutefont{LGR}{\rmdefault}{cmr}
  \substitutefont{LGR}{\sfdefault}{cmss}
  \substitutefont{LGR}{\ttdefault}{cmtt}
\fi
\expandafter\ifx\csname T@X2\endcsname\relax
  \expandafter\ifx\csname T@T2A\endcsname\relax
  \else
  % T2A was declared as font encoding
    \substitutefont{T2A}{\rmdefault}{cmr}
    \substitutefont{T2A}{\sfdefault}{cmss}
    \substitutefont{T2A}{\ttdefault}{cmtt}
  \fi
\else
% X2 was declared as font encoding
  \substitutefont{X2}{\rmdefault}{cmr}
  \substitutefont{X2}{\sfdefault}{cmss}
  \substitutefont{X2}{\ttdefault}{cmtt}
\fi


\usepackage[Sonny]{fncychap}
\ChNameVar{\Large\normalfont\sffamily}
\ChTitleVar{\Large\normalfont\sffamily}
\usepackage{sphinx}

\fvset{fontsize=\small}
\usepackage{geometry}


% Include hyperref last.
\usepackage{hyperref}
% Fix anchor placement for figures with captions.
\usepackage{hypcap}% it must be loaded after hyperref.
% Set up styles of URL: it should be placed after hyperref.
\urlstyle{same}


\usepackage{sphinxmessages}
\setcounter{tocdepth}{1}



\title{GRID}
\date{avr. 25, 2021}
\release{1.0}
\author{GRID Team}
\newcommand{\sphinxlogo}{\vbox{}}
\renewcommand{\releasename}{Version}
\makeindex
\begin{document}

\ifdefined\shorthandoff
  \ifnum\catcode`\=\string=\active\shorthandoff{=}\fi
  \ifnum\catcode`\"=\active\shorthandoff{"}\fi
\fi

\pagestyle{empty}
\sphinxmaketitle
\pagestyle{plain}
\sphinxtableofcontents
\pagestyle{normal}
\phantomsection\label{\detokenize{index::doc}}


\sphinxAtStartPar
Bienvenue sur la documentation du GRID!

\sphinxAtStartPar
Ici, vous trouverez:
\begin{enumerate}
\sphinxsetlistlabels{\arabic}{enumi}{enumii}{}{.}%
\item {} 
\sphinxAtStartPar
Le fonctionement général du projet

\item {} 
\sphinxAtStartPar
L’insfrastructure du projet

\item {} 
\sphinxAtStartPar
Comment les indicateurs sont calculés

\item {} 
\sphinxAtStartPar
Comment lire le projet

\item {} 
\sphinxAtStartPar
La documentation spécifique au moudule \sphinxtitleref{data\_agri} qui génère les données

\item {} 
\sphinxAtStartPar
La documentation spécifique au module \sphinxtitleref{app} qui génère l’app

\end{enumerate}

\sphinxAtStartPar
\sphinxhref{http://docs.grid-tech.fr/grid.pdf}{Documentation PDF}

\begin{sphinxadmonition}{attention}{Attention:}
\sphinxAtStartPar
Le dashboard est pour le moment optimisé uniquement pour les écrans d’ordinateurs
\end{sphinxadmonition}


\bigskip\hrule\bigskip



\chapter{Table des matières}
\label{\detokenize{index:table-des-matieres}}

\section{Introduction}
\label{\detokenize{intro:introduction}}\label{\detokenize{intro::doc}}

\subsection{Se connecter}
\label{\detokenize{intro:se-connecter}}
\sphinxAtStartPar
Pour voir le dashboard, rendez\sphinxhyphen{}vous sur \sphinxhref{http://app.grid-tech.fr}{app.grid\sphinxhyphen{}tech.fr} .

\sphinxAtStartPar
Pour les besoins de la démonstration dans le cadre du concours FIFG, un compte test a été créé :
\begin{itemize}
\item {} 
\sphinxAtStartPar
\sphinxstylestrong{Nom d’utilisateur}: test

\item {} 
\sphinxAtStartPar
\sphinxstylestrong{Mot de passe}: test

\end{itemize}

\begin{sphinxadmonition}{tip}{Astuce:}
\sphinxAtStartPar
Avec notre infrastructure actuelle, le chargement des pages peut paraître long mais nous travaillons pour y remédier.
\end{sphinxadmonition}

\begin{sphinxadmonition}{attention}{Attention:}
\sphinxAtStartPar
Le dashboard est pour le moment optimisé pour les écrans d’ordinateurs.
\end{sphinxadmonition}

\sphinxAtStartPar
Pour plus de détails sur le déploiement {\hyperref[\detokenize{use:instal}]{\sphinxcrossref{\DUrole{std,std-ref}{Déploiement et installation}}}}


\subsection{Les données}
\label{\detokenize{intro:les-donnees}}

\subsubsection{Les données d’entrée}
\label{\detokenize{intro:les-donnees-d-entree}}
\sphinxAtStartPar
Le GRID fonctionne à partir de 2 types de données d’entrée:
\begin{itemize}
\item {} 
\sphinxAtStartPar
Les données externes provenant de Météo France, Copernicus, etc ;

\item {} \begin{description}
\item[{Les données liées à l’exploitation :}] \leavevmode\begin{itemize}
\item {} 
\sphinxAtStartPar
Données internes rentrées par l’agriculteur dans un questionnaire ;

\item {} 
\sphinxAtStartPar
Données financières provenant de l’établissement bancaire.

\end{itemize}

\end{description}

\end{itemize}


\subsubsection{Données pour le PoC}
\label{\detokenize{intro:donnees-pour-le-poc}}
\sphinxAtStartPar
Pour le PoC, afin de démontrer la capacité dynamique du dashboard, à chaque login, une partie des données sont tirées au hasard, en particulier celles relatives :
\begin{itemize}
\item {} 
\sphinxAtStartPar
aux scores RSE présentés sur la première page

\item {} 
\sphinxAtStartPar
aux données financières (cf page indicateur et le module \sphinxtitleref{agri\_data} pour plus de détail)

\end{itemize}

\sphinxAtStartPar
Toutes les données sont disponibles \sphinxhref{https://github.com/Green-Investement-Dashboard/data/tree/main/data\_eg}{ici}

\sphinxAtStartPar
Pour plus de détail sur les données et leur exploitation {\hyperref[\detokenize{indicateurs:indic}]{\sphinxcrossref{\DUrole{std,std-ref}{Indicateurs et graphiques}}}}


\subsection{Le code}
\label{\detokenize{intro:le-code}}
\sphinxAtStartPar
Vous pouvez retrouver notre code dans son intégralité \sphinxhref{https://github.com/Green-Investement-Dashboard/GRID\_app}{ici}.


\section{Infrastructure}
\label{\detokenize{infrastructure:infrastructure}}\label{\detokenize{infrastructure::doc}}

\subsection{Organisation globale du répertoire}
\label{\detokenize{infrastructure:organisation-globale-du-repertoire}}
\sphinxAtStartPar
Le répertoire est organisé autour de deux principaux modules :
\begin{itemize}
\item {} 
\sphinxAtStartPar
\sphinxcode{\sphinxupquote{app}} : module générant l’application et ses rendus

\item {} 
\sphinxAtStartPar
\sphinxcode{\sphinxupquote{agri\_data}} : module regroupant et générant les données

\end{itemize}

\sphinxAtStartPar
Le fichier principal qui lance l’application est situé à la racine, il s’appelle \sphinxcode{\sphinxupquote{run.py}}

\begin{sphinxVerbatim}[commandchars=\\\{\}]
\PYG{p}{|}
\PYG{p}{|}\PYGZhy{}\PYGZhy{} app/                                      \PYG{c+c1}{\PYGZsh{} L\PYGZsq{}application en elle même}
\PYG{p}{|}    \PYG{p}{|}\PYGZhy{}\PYGZhy{} home/                                \PYG{c+c1}{\PYGZsh{} Génération des contenus spécifiques pas page HTML spécifiques}
\PYG{p}{|}    \PYG{p}{|}\PYGZhy{}\PYGZhy{} base/                                \PYG{c+c1}{\PYGZsh{} Blueprint, contient la structure de l\PYGZsq{}application}
\PYG{p}{|}
\PYG{p}{|}\PYGZhy{}\PYGZhy{} agri\PYGZus{}data/                                \PYG{c+c1}{\PYGZsh{} Génération des données}
\PYG{p}{|}
\PYG{p}{|}\PYGZhy{}\PYGZhy{} ************************************************************************
\end{sphinxVerbatim}


\subsection{Point de vue Flask App / Python}
\label{\detokenize{infrastructure:point-de-vue-flask-app-python}}
\sphinxAtStartPar
Le module \sphinxcode{\sphinxupquote{app}} est organisé en 2 sous\sphinxhyphen{}modules :
\begin{itemize}
\item {} 
\sphinxAtStartPar
\sphinxcode{\sphinxupquote{home}} qui sert à générer les visuels, en particulier le sous\sphinxhyphen{}module \sphinxcode{\sphinxupquote{content\_gen}}

\item {} 
\sphinxAtStartPar
\sphinxcode{\sphinxupquote{base}} qui sert à gérer l’authentification

\end{itemize}

\sphinxAtStartPar
Le code est ensuite commenté et précisé dans les modules {\hyperref[\detokenize{modules:agri}]{\sphinxcrossref{\DUrole{std,std-ref}{Module agri\_data}}}} et {\hyperref[\detokenize{app:app}]{\sphinxcrossref{\DUrole{std,std-ref}{Module app}}}}

\sphinxAtStartPar
La structure du répértoire ,d’un point de vue Python, est la suivante:

\begin{sphinxVerbatim}[commandchars=\\\{\}]
\PYG{p}{|}
\PYG{p}{|}\PYGZhy{}\PYGZhy{} app/                                      \PYG{c+c1}{\PYGZsh{} L\PYGZsq{}application en elle même}
\PYG{p}{|}    \PYG{p}{|}\PYGZhy{}\PYGZhy{} home/
\PYG{p}{|}         \PYG{p}{|}\PYGZhy{}\PYGZhy{} content\PYGZus{}gen/                    \PYG{c+c1}{\PYGZsh{} Module générant les visuels}
\PYG{p}{|}         \PYG{p}{|}    \PYG{p}{|}\PYGZhy{}\PYGZhy{} data/                      \PYG{c+c1}{\PYGZsh{} Données externes pré\PYGZhy{}traitées}
\PYG{p}{|}         \PYG{p}{|}    \PYG{p}{|}\PYGZhy{}\PYGZhy{} graph\PYGZus{}generation.py        \PYG{c+c1}{\PYGZsh{} Génération des graphiques}
\PYG{p}{|}         \PYG{p}{|}    \PYG{p}{|}\PYGZhy{}\PYGZhy{} index\PYGZus{}renderer.py          \PYG{c+c1}{\PYGZsh{} Génération de l\PYGZsq{}index}
\PYG{p}{|}         \PYG{p}{|}    \PYG{p}{|}\PYGZhy{}\PYGZhy{} map\PYGZus{}generation.py          \PYG{c+c1}{\PYGZsh{} Génération des cartes}
\PYG{p}{|}         \PYG{p}{|}    \PYG{p}{|}\PYGZhy{}\PYGZhy{} questionaire.py.py         \PYG{c+c1}{\PYGZsh{} Génération du questionnaire agri}
\PYG{p}{|}         \PYG{p}{|}\PYGZhy{}\PYGZhy{} routes.py
\PYG{p}{|}
\PYG{p}{|}    \PYG{p}{|}\PYGZhy{}\PYGZhy{} base/
\PYG{p}{|}         \PYG{p}{|}\PYGZhy{}\PYGZhy{} forms.py                        \PYG{c+c1}{\PYGZsh{} Script gérant le formulaire de login et d\PYGZsq{}inscription}
\PYG{p}{|}         \PYG{p}{|}\PYGZhy{}\PYGZhy{} models.py                       \PYG{c+c1}{\PYGZsh{} Script gérant la lecture de la base de données des logins}
\PYG{p}{|}         \PYG{p}{|}\PYGZhy{}\PYGZhy{} routes.py                       \PYG{c+c1}{\PYGZsh{} Script gérant les actions}
\PYG{p}{|}         \PYG{p}{|}\PYGZhy{}\PYGZhy{} util.py                         \PYG{c+c1}{\PYGZsh{} Script gérant le hachage du mot de passe}
\PYG{p}{|}
\PYG{p}{|}\PYGZhy{}\PYGZhy{} agri\PYGZus{}data/
\PYG{p}{|}    \PYG{p}{|}\PYGZhy{}\PYGZhy{} data\PYGZus{}draw.py                         \PYG{c+c1}{\PYGZsh{} Tirage aléatoire des données}
\PYG{p}{|}    \PYG{p}{|}\PYGZhy{}\PYGZhy{} data\PYGZus{}import.py                       \PYG{c+c1}{\PYGZsh{} Import des données de GitHub}
\PYG{p}{|}    \PYG{p}{|}\PYGZhy{}\PYGZhy{} *.json
\PYG{p}{|}
\PYG{p}{|}\PYGZhy{}\PYGZhy{} requirements.txt                          \PYG{c+c1}{\PYGZsh{} Librairies nécessaires pour faire fonctionner le code}
\PYG{p}{|}\PYGZhy{}\PYGZhy{} environment.yml                           \PYG{c+c1}{\PYGZsh{} Environnement anaconda}
\PYG{p}{|}\PYGZhy{}\PYGZhy{} requirements\PYGZhy{}mysql.txt                    \PYG{c+c1}{\PYGZsh{} Module nécessaire pour Mysql DMBS}
\PYG{p}{|}\PYGZhy{}\PYGZhy{} requirements\PYGZhy{}pqsql.txt                    \PYG{c+c1}{\PYGZsh{} Module nécessaire pour PostgreSql DMBS}
\PYG{p}{|}
\PYG{p}{|}\PYGZhy{}\PYGZhy{} .env                                      \PYG{c+c1}{\PYGZsh{} Variable environnement}
\PYG{p}{|}\PYGZhy{}\PYGZhy{} config.py                                 \PYG{c+c1}{\PYGZsh{} Configuration de l\PYGZsq{}application}
\PYG{p}{|}\PYGZhy{}\PYGZhy{} run.py                                    \PYG{c+c1}{\PYGZsh{} Lancement de l\PYGZsq{}application}
\PYG{p}{|}
\PYG{p}{|}\PYGZhy{}\PYGZhy{} ************************************************************************
\end{sphinxVerbatim}


\subsection{Point de vue fronte\sphinxhyphen{}end / HTML}
\label{\detokenize{infrastructure:point-de-vue-fronte-end-html}}
\sphinxAtStartPar
Les fichiers HTML sont organisés autour de 2 dossiers:
\begin{itemize}
\item {} 
\sphinxAtStartPar
\sphinxstylestrong{/home} : ici sont stockés les fichiers HTML des pages du dashboard

\item {} 
\sphinxAtStartPar
\sphinxstylestrong{/base} : ici sont stockés les fichiers HTML servant de modèles pour générer les pages

\end{itemize}

\begin{sphinxVerbatim}[commandchars=\\\{\}]
\PYG{p}{|}
\PYG{p}{|}\PYGZhy{}\PYGZhy{} app/
\PYG{p}{|}    \PYG{p}{|}\PYGZhy{}\PYGZhy{} home/
\PYG{p}{|}         \PYG{p}{|}\PYGZhy{}\PYGZhy{} templates/                       \PYG{c+c1}{\PYGZsh{} Ensemble des pages HTML}
\PYG{p}{|}         \PYG{p}{|}    \PYG{p}{|}    \PYG{p}{|}\PYGZhy{}\PYGZhy{} *.html
\PYG{p}{|}
\PYG{p}{|}    \PYG{p}{|}\PYGZhy{}\PYGZhy{} base/
\PYG{p}{|}         \PYG{p}{|}\PYGZhy{}\PYGZhy{} static/
\PYG{p}{|}         \PYG{p}{|}    \PYG{p}{|}\PYGZhy{}\PYGZhy{} \PYGZlt{}css, JS, images\PYGZgt{}          \PYG{c+c1}{\PYGZsh{} Fichiers CSS, Javascripts et images}
\PYG{p}{|}         \PYG{p}{|}
\PYG{p}{|}         \PYG{p}{|}\PYGZhy{}\PYGZhy{} templates/                      \PYG{c+c1}{\PYGZsh{} Modèles pour le rendu des pages}
\PYG{p}{|}              \PYG{p}{|}
\PYG{p}{|}              \PYG{p}{|}\PYGZhy{}\PYGZhy{} includes/
\PYG{p}{|}              \PYG{p}{|}    \PYG{p}{|}\PYGZhy{}\PYGZhy{} navigation.html       \PYG{c+c1}{\PYGZsh{} Menu du haut}
\PYG{p}{|}              \PYG{p}{|}    \PYG{p}{|}\PYGZhy{}\PYGZhy{} sidebar.html          \PYG{c+c1}{\PYGZsh{} Menu latéral}
\PYG{p}{|}              \PYG{p}{|}    \PYG{p}{|}\PYGZhy{}\PYGZhy{} footer.html           \PYG{c+c1}{\PYGZsh{} Pied de page}
\PYG{p}{|}              \PYG{p}{|}    \PYG{p}{|}\PYGZhy{}\PYGZhy{} scripts.html          \PYG{c+c1}{\PYGZsh{} Scripts communs aux pages HTML}
\PYG{p}{|}              \PYG{p}{|}
\PYG{p}{|}              \PYG{p}{|}\PYGZhy{}\PYGZhy{} layouts/                   \PYG{c+c1}{\PYGZsh{} Pages masters}
\PYG{p}{|}              \PYG{p}{|}    \PYG{p}{|}\PYGZhy{}\PYGZhy{} base.html             \PYG{c+c1}{\PYGZsh{} Layout des pages}
\PYG{p}{|}              \PYG{p}{|}
\PYG{p}{|}              \PYG{p}{|}\PYGZhy{}\PYGZhy{} accounts/                  \PYG{c+c1}{\PYGZsh{} Pages authentification}
\PYG{p}{|}                   \PYG{p}{|}\PYGZhy{}\PYGZhy{} login.html            \PYG{c+c1}{\PYGZsh{} Page de Login}
\PYG{p}{|}                   \PYG{p}{|}\PYGZhy{}\PYGZhy{} register.html         \PYG{c+c1}{\PYGZsh{} Page d\PYGZsq{}inscription}
\PYG{p}{|}
\PYG{p}{|}\PYGZhy{}\PYGZhy{} ************************************************************************
\end{sphinxVerbatim}


\section{Indicateurs et graphiques}
\label{\detokenize{indicateurs:indicateurs-et-graphiques}}\label{\detokenize{indicateurs:indic}}\label{\detokenize{indicateurs::doc}}

\subsection{Les types de représentations}
\label{\detokenize{indicateurs:les-types-de-representations}}
\sphinxAtStartPar
Afin de rendre compte au mieux des données, nous utilisons trois types de représentations:
\begin{itemize}
\item {} 
\sphinxAtStartPar
\sphinxstylestrong{Compteurs}: ceux\sphinxhyphen{}ci codés en JS représentent les 3 scores ESG sur la page d’accueil.

\item {} 
\sphinxAtStartPar
\sphinxstylestrong{Graphiques}: que ce soit des graphiques lignes ou à barres ils servent à représenter l’évolution temporelle d’un indicateur .

\item {} 
\sphinxAtStartPar
\sphinxstylestrong{Echelles de couleurs}: lorsque qu’un indicateur est calculé à partir d’un modèle, il est représenté sous la forme d’une échelle de couleurs comme on peut le retrouver dans l’onglet Social avec le rayonnement de l’exploitation.

\item {} 
\sphinxAtStartPar
\sphinxstylestrong{Cartes}: ce support est utilisé pour représenter des données spatiales avec une dimension temporelle.

\end{itemize}


\subsection{Exemple d’indicateurs}
\label{\detokenize{indicateurs:exemple-d-indicateurs}}

\subsubsection{Carte des feu de forêts}
\label{\detokenize{indicateurs:carte-des-feu-de-forets}}
\sphinxAtStartPar
Sur la base des données \sphinxhref{https://cds.climate.copernicus.eu/cdsapp\#!/dataset/sis-tourism-fire-danger-indicators?tab=overview}{du Climate Data Store}, base de données de l’UE, nous avons pu exporter ces données, les traiter et les nettoyer pour notre usage. Nous avons décidé de choisir les données du modèle du GIEC RCP 4.5 car celui\sphinxhyphen{}ci correspond au scenario le plus probable.
Ces données ont ensuite été présentées sur une carte disponible dans l’onglet Environnement.


\subsubsection{Graph des canicules}
\label{\detokenize{indicateurs:graph-des-canicules}}
\sphinxAtStartPar
Toujours sur la base des données \sphinxhref{https://cds.climate.copernicus.eu/cdsapp\#!/dataset/sis-heat-and-cold-spells?tab=overview}{du Climate Data Store}, nous avons selectionner ces données représentant le nombre de jours de canicule. Il nous est paru plus pertinent de représenter les jours de canicule uniquement à l’emplacement du viticultuteur.


\section{Déploiement et installation}
\label{\detokenize{use:deploiement-et-installation}}\label{\detokenize{use:instal}}\label{\detokenize{use::doc}}

\subsection{Déploiement en ligne}
\label{\detokenize{use:deploiement-en-ligne}}
\sphinxAtStartPar
Le code est stocké sur GitHub puis déployé sur Heroku pour qu’il soit accessible en ligne. Ce choix a été fait pour simplifier la création et la visualtions du PoC dans un premier temps. Cependant, à terme, l’application sera hébergée sur Google Cloud.

\sphinxAtStartPar
La principale conséquence de ce choix est le temps que met l’application à charger.


\subsection{Installation en local}
\label{\detokenize{use:installation-en-local}}
\sphinxAtStartPar
Si vous le souhaitez, il est possible de faire tourner l’application en local, cependant cela nécessite Python 3.x et un manager de module type pip ou anaconda.
Pour la suite, nous supposerons que ces pré\sphinxhyphen{}requis sont remplis.

\sphinxAtStartPar
Pour utiliser l’application en local:
\begin{enumerate}
\sphinxsetlistlabels{\arabic}{enumi}{enumii}{}{.}%
\item {} 
\sphinxAtStartPar
Clonez la branche principale du \sphinxhref{https://github.com/Green-Investement-Dashboard/GRID\_app}{répértoire GitHub}

\item {} \begin{description}
\item[{Créez un environnement virtuel soit avec :}] \leavevmode\begin{enumerate}
\sphinxsetlistlabels{\arabic}{enumii}{enumiii}{}{.}%
\item {} 
\sphinxAtStartPar
pip : \sphinxcode{\sphinxupquote{python3 \sphinxhyphen{}m pip install \sphinxhyphen{}r requirements.txt}}

\item {} 
\sphinxAtStartPar
anaconda \sphinxcode{\sphinxupquote{conda env create \sphinxhyphen{}f environment.yml}}

\end{enumerate}

\end{description}

\end{enumerate}


\section{Module agri\_data}
\label{\detokenize{modules:module-agri-data}}\label{\detokenize{modules:agri}}\label{\detokenize{modules::doc}}

\subsection{agri\_data package}
\label{\detokenize{agri_data:agri-data-package}}\label{\detokenize{agri_data::doc}}

\subsubsection{Submodules}
\label{\detokenize{agri_data:submodules}}

\subsubsection{agri\_data.data\_draw module}
\label{\detokenize{agri_data:module-agri_data.data_draw}}\label{\detokenize{agri_data:agri-data-data-draw-module}}\index{module@\spxentry{module}!agri\_data.data\_draw@\spxentry{agri\_data.data\_draw}}\index{agri\_data.data\_draw@\spxentry{agri\_data.data\_draw}!module@\spxentry{module}}
\sphinxAtStartPar
© GRID Team, 2021
\index{RandomDraw (classe dans agri\_data.data\_draw)@\spxentry{RandomDraw}\spxextra{classe dans agri\_data.data\_draw}}

\begin{fulllineitems}
\phantomsection\label{\detokenize{agri_data:agri_data.data_draw.RandomDraw}}\pysigline{\sphinxbfcode{\sphinxupquote{class }}\sphinxcode{\sphinxupquote{agri\_data.data\_draw.}}\sphinxbfcode{\sphinxupquote{RandomDraw}}}
\sphinxAtStartPar
Bases : \sphinxcode{\sphinxupquote{object}}

\sphinxAtStartPar
Cette classe télécharge les données de GitHub et les stocke en local. Pour certains jeux de données, ils sont modifiés par un tri
alétoire à chaque login
\index{data\_agri() (méthode agri\_data.data\_draw.RandomDraw)@\spxentry{data\_agri()}\spxextra{méthode agri\_data.data\_draw.RandomDraw}}

\begin{fulllineitems}
\phantomsection\label{\detokenize{agri_data:agri_data.data_draw.RandomDraw.data_agri}}\pysiglinewithargsret{\sphinxbfcode{\sphinxupquote{data\_agri}}}{}{}
\sphinxAtStartPar
Télécharge et enregistre les données liées à l’emplacement de l’agriculteur.

\end{fulllineitems}

\index{financial\_data() (méthode agri\_data.data\_draw.RandomDraw)@\spxentry{financial\_data()}\spxextra{méthode agri\_data.data\_draw.RandomDraw}}

\begin{fulllineitems}
\phantomsection\label{\detokenize{agri_data:agri_data.data_draw.RandomDraw.financial_data}}\pysiglinewithargsret{\sphinxbfcode{\sphinxupquote{financial\_data}}}{}{}
\sphinxAtStartPar
Télécharge et enregistre les données liées aux données financières.
Elles sont randomisées avant l’enregistrement.

\end{fulllineitems}

\index{gauges\_val() (méthode agri\_data.data\_draw.RandomDraw)@\spxentry{gauges\_val()}\spxextra{méthode agri\_data.data\_draw.RandomDraw}}

\begin{fulllineitems}
\phantomsection\label{\detokenize{agri_data:agri_data.data_draw.RandomDraw.gauges_val}}\pysiglinewithargsret{\sphinxbfcode{\sphinxupquote{gauges\_val}}}{}{}
\sphinxAtStartPar
Télécharge et enregistre les données pour générer les échelles de couleurs.

\end{fulllineitems}

\index{graph\_val() (méthode agri\_data.data\_draw.RandomDraw)@\spxentry{graph\_val()}\spxextra{méthode agri\_data.data\_draw.RandomDraw}}

\begin{fulllineitems}
\phantomsection\label{\detokenize{agri_data:agri_data.data_draw.RandomDraw.graph_val}}\pysiglinewithargsret{\sphinxbfcode{\sphinxupquote{graph\_val}}}{}{}
\sphinxAtStartPar
Télécharge et enregistre les données pour générer les graphs.

\end{fulllineitems}

\index{indic\_critique() (méthode agri\_data.data\_draw.RandomDraw)@\spxentry{indic\_critique()}\spxextra{méthode agri\_data.data\_draw.RandomDraw}}

\begin{fulllineitems}
\phantomsection\label{\detokenize{agri_data:agri_data.data_draw.RandomDraw.indic_critique}}\pysiglinewithargsret{\sphinxbfcode{\sphinxupquote{indic\_critique}}}{}{}
\sphinxAtStartPar
Télécharge et enregistre les données donnant les indices critiques.

\end{fulllineitems}

\index{main() (méthode agri\_data.data\_draw.RandomDraw)@\spxentry{main()}\spxextra{méthode agri\_data.data\_draw.RandomDraw}}

\begin{fulllineitems}
\phantomsection\label{\detokenize{agri_data:agri_data.data_draw.RandomDraw.main}}\pysiglinewithargsret{\sphinxbfcode{\sphinxupquote{main}}}{}{}
\end{fulllineitems}

\index{scoring\_data() (méthode agri\_data.data\_draw.RandomDraw)@\spxentry{scoring\_data()}\spxextra{méthode agri\_data.data\_draw.RandomDraw}}

\begin{fulllineitems}
\phantomsection\label{\detokenize{agri_data:agri_data.data_draw.RandomDraw.scoring_data}}\pysiglinewithargsret{\sphinxbfcode{\sphinxupquote{scoring\_data}}}{}{}
\sphinxAtStartPar
Télécharge et enregistre les données de scoring RSE.
Elles sont randomisées avant l’enregistrement.

\end{fulllineitems}

\index{stat\_data() (méthode agri\_data.data\_draw.RandomDraw)@\spxentry{stat\_data()}\spxextra{méthode agri\_data.data\_draw.RandomDraw}}

\begin{fulllineitems}
\phantomsection\label{\detokenize{agri_data:agri_data.data_draw.RandomDraw.stat_data}}\pysiglinewithargsret{\sphinxbfcode{\sphinxupquote{stat\_data}}}{}{}
\sphinxAtStartPar
Télécharge et enregistre les données donnant les statiques liés à la région.

\end{fulllineitems}


\end{fulllineitems}



\subsubsection{agri\_data.data\_import module}
\label{\detokenize{agri_data:module-agri_data.data_import}}\label{\detokenize{agri_data:agri-data-data-import-module}}\index{module@\spxentry{module}!agri\_data.data\_import@\spxentry{agri\_data.data\_import}}\index{agri\_data.data\_import@\spxentry{agri\_data.data\_import}!module@\spxentry{module}}
\sphinxAtStartPar
© GRID Team, 2021
\index{ReadData (classe dans agri\_data.data\_import)@\spxentry{ReadData}\spxextra{classe dans agri\_data.data\_import}}

\begin{fulllineitems}
\phantomsection\label{\detokenize{agri_data:agri_data.data_import.ReadData}}\pysiglinewithargsret{\sphinxbfcode{\sphinxupquote{class }}\sphinxcode{\sphinxupquote{agri\_data.data\_import.}}\sphinxbfcode{\sphinxupquote{ReadData}}}{\emph{\DUrole{n}{name}}}{}
\sphinxAtStartPar
Bases : \sphinxcode{\sphinxupquote{object}}

\sphinxAtStartPar
Cette classe lit les données json disponibles en locals et retourne une dataframe
\index{read\_json() (méthode agri\_data.data\_import.ReadData)@\spxentry{read\_json()}\spxextra{méthode agri\_data.data\_import.ReadData}}

\begin{fulllineitems}
\phantomsection\label{\detokenize{agri_data:agri_data.data_import.ReadData.read_json}}\pysiglinewithargsret{\sphinxbfcode{\sphinxupquote{read\_json}}}{}{}
\end{fulllineitems}


\end{fulllineitems}



\subsubsection{Module contents}
\label{\detokenize{agri_data:module-agri_data}}\label{\detokenize{agri_data:module-contents}}\index{module@\spxentry{module}!agri\_data@\spxentry{agri\_data}}\index{agri\_data@\spxentry{agri\_data}!module@\spxentry{module}}

\section{Module app}
\label{\detokenize{app:module-app}}\label{\detokenize{app:app}}\label{\detokenize{app::doc}}

\subsection{Subpackages}
\label{\detokenize{app:subpackages}}

\subsubsection{app.base package}
\label{\detokenize{app.base:app-base-package}}\label{\detokenize{app.base::doc}}

\paragraph{Submodules}
\label{\detokenize{app.base:submodules}}

\paragraph{app.base.forms module}
\label{\detokenize{app.base:module-app.base.forms}}\label{\detokenize{app.base:app-base-forms-module}}\index{module@\spxentry{module}!app.base.forms@\spxentry{app.base.forms}}\index{app.base.forms@\spxentry{app.base.forms}!module@\spxentry{module}}
\sphinxAtStartPar
Modified for GRID, 2021

\sphinxAtStartPar
Copyright (c) 2019 \sphinxhyphen{} present AppSeed.us

\sphinxAtStartPar
Génère les formulaires d’inscription et de connexion
\index{CreateAccountForm (classe dans app.base.forms)@\spxentry{CreateAccountForm}\spxextra{classe dans app.base.forms}}

\begin{fulllineitems}
\phantomsection\label{\detokenize{app.base:app.base.forms.CreateAccountForm}}\pysiglinewithargsret{\sphinxbfcode{\sphinxupquote{class }}\sphinxcode{\sphinxupquote{app.base.forms.}}\sphinxbfcode{\sphinxupquote{CreateAccountForm}}}{\emph{\DUrole{o}{*}\DUrole{n}{args}}, \emph{\DUrole{o}{**}\DUrole{n}{kwargs}}}{}
\sphinxAtStartPar
Bases : \sphinxcode{\sphinxupquote{flask\_wtf.form.FlaskForm}}
\index{email (attribut app.base.forms.CreateAccountForm)@\spxentry{email}\spxextra{attribut app.base.forms.CreateAccountForm}}

\begin{fulllineitems}
\phantomsection\label{\detokenize{app.base:app.base.forms.CreateAccountForm.email}}\pysigline{\sphinxbfcode{\sphinxupquote{email}}\sphinxbfcode{\sphinxupquote{ = \textless{}UnboundField(TextField, (\textquotesingle{}Email\textquotesingle{},), \{\textquotesingle{}id\textquotesingle{}: \textquotesingle{}email\_create\textquotesingle{}, \textquotesingle{}validators\textquotesingle{}: {[}\textless{}wtforms.validators.DataRequired object\textgreater{}, \textless{}wtforms.validators.Email object\textgreater{}{]}\})\textgreater{}}}}
\end{fulllineitems}

\index{password (attribut app.base.forms.CreateAccountForm)@\spxentry{password}\spxextra{attribut app.base.forms.CreateAccountForm}}

\begin{fulllineitems}
\phantomsection\label{\detokenize{app.base:app.base.forms.CreateAccountForm.password}}\pysigline{\sphinxbfcode{\sphinxupquote{password}}\sphinxbfcode{\sphinxupquote{ = \textless{}UnboundField(PasswordField, (\textquotesingle{}Password\textquotesingle{},), \{\textquotesingle{}id\textquotesingle{}: \textquotesingle{}pwd\_create\textquotesingle{}, \textquotesingle{}validators\textquotesingle{}: {[}\textless{}wtforms.validators.DataRequired object\textgreater{}{]}\})\textgreater{}}}}
\end{fulllineitems}

\index{username (attribut app.base.forms.CreateAccountForm)@\spxentry{username}\spxextra{attribut app.base.forms.CreateAccountForm}}

\begin{fulllineitems}
\phantomsection\label{\detokenize{app.base:app.base.forms.CreateAccountForm.username}}\pysigline{\sphinxbfcode{\sphinxupquote{username}}\sphinxbfcode{\sphinxupquote{ = \textless{}UnboundField(TextField, (\textquotesingle{}Username\textquotesingle{},), \{\textquotesingle{}id\textquotesingle{}: \textquotesingle{}username\_create\textquotesingle{}, \textquotesingle{}validators\textquotesingle{}: {[}\textless{}wtforms.validators.DataRequired object\textgreater{}{]}\})\textgreater{}}}}
\end{fulllineitems}


\end{fulllineitems}

\index{LoginForm (classe dans app.base.forms)@\spxentry{LoginForm}\spxextra{classe dans app.base.forms}}

\begin{fulllineitems}
\phantomsection\label{\detokenize{app.base:app.base.forms.LoginForm}}\pysiglinewithargsret{\sphinxbfcode{\sphinxupquote{class }}\sphinxcode{\sphinxupquote{app.base.forms.}}\sphinxbfcode{\sphinxupquote{LoginForm}}}{\emph{\DUrole{o}{*}\DUrole{n}{args}}, \emph{\DUrole{o}{**}\DUrole{n}{kwargs}}}{}
\sphinxAtStartPar
Bases : \sphinxcode{\sphinxupquote{flask\_wtf.form.FlaskForm}}
\index{password (attribut app.base.forms.LoginForm)@\spxentry{password}\spxextra{attribut app.base.forms.LoginForm}}

\begin{fulllineitems}
\phantomsection\label{\detokenize{app.base:app.base.forms.LoginForm.password}}\pysigline{\sphinxbfcode{\sphinxupquote{password}}\sphinxbfcode{\sphinxupquote{ = \textless{}UnboundField(PasswordField, (\textquotesingle{}Password\textquotesingle{},), \{\textquotesingle{}id\textquotesingle{}: \textquotesingle{}pwd\_login\textquotesingle{}, \textquotesingle{}validators\textquotesingle{}: {[}\textless{}wtforms.validators.DataRequired object\textgreater{}{]}\})\textgreater{}}}}
\end{fulllineitems}

\index{username (attribut app.base.forms.LoginForm)@\spxentry{username}\spxextra{attribut app.base.forms.LoginForm}}

\begin{fulllineitems}
\phantomsection\label{\detokenize{app.base:app.base.forms.LoginForm.username}}\pysigline{\sphinxbfcode{\sphinxupquote{username}}\sphinxbfcode{\sphinxupquote{ = \textless{}UnboundField(TextField, (\textquotesingle{}Username\textquotesingle{},), \{\textquotesingle{}id\textquotesingle{}: \textquotesingle{}username\_login\textquotesingle{}, \textquotesingle{}validators\textquotesingle{}: {[}\textless{}wtforms.validators.DataRequired object\textgreater{}{]}\})\textgreater{}}}}
\end{fulllineitems}


\end{fulllineitems}



\paragraph{app.base.models module}
\label{\detokenize{app.base:module-app.base.models}}\label{\detokenize{app.base:app-base-models-module}}\index{module@\spxentry{module}!app.base.models@\spxentry{app.base.models}}\index{app.base.models@\spxentry{app.base.models}!module@\spxentry{module}}
\sphinxAtStartPar
Modified for GRID, 2021

\sphinxAtStartPar
Copyright (c) 2019 \sphinxhyphen{} present AppSeed.us

\sphinxAtStartPar
Sert à lire et écrire dans la db des logins
\index{User (classe dans app.base.models)@\spxentry{User}\spxextra{classe dans app.base.models}}

\begin{fulllineitems}
\phantomsection\label{\detokenize{app.base:app.base.models.User}}\pysiglinewithargsret{\sphinxbfcode{\sphinxupquote{class }}\sphinxcode{\sphinxupquote{app.base.models.}}\sphinxbfcode{\sphinxupquote{User}}}{\emph{\DUrole{o}{**}\DUrole{n}{kwargs}}}{}
\sphinxAtStartPar
Bases : \sphinxcode{\sphinxupquote{sqlalchemy.ext.declarative.api.Model}}, \sphinxcode{\sphinxupquote{flask\_login.mixins.UserMixin}}
\index{email (attribut app.base.models.User)@\spxentry{email}\spxextra{attribut app.base.models.User}}

\begin{fulllineitems}
\phantomsection\label{\detokenize{app.base:app.base.models.User.email}}\pysigline{\sphinxbfcode{\sphinxupquote{email}}}
\end{fulllineitems}

\index{id (attribut app.base.models.User)@\spxentry{id}\spxextra{attribut app.base.models.User}}

\begin{fulllineitems}
\phantomsection\label{\detokenize{app.base:app.base.models.User.id}}\pysigline{\sphinxbfcode{\sphinxupquote{id}}}
\end{fulllineitems}

\index{password (attribut app.base.models.User)@\spxentry{password}\spxextra{attribut app.base.models.User}}

\begin{fulllineitems}
\phantomsection\label{\detokenize{app.base:app.base.models.User.password}}\pysigline{\sphinxbfcode{\sphinxupquote{password}}}
\end{fulllineitems}

\index{username (attribut app.base.models.User)@\spxentry{username}\spxextra{attribut app.base.models.User}}

\begin{fulllineitems}
\phantomsection\label{\detokenize{app.base:app.base.models.User.username}}\pysigline{\sphinxbfcode{\sphinxupquote{username}}}
\end{fulllineitems}


\end{fulllineitems}

\index{request\_loader() (dans le module app.base.models)@\spxentry{request\_loader()}\spxextra{dans le module app.base.models}}

\begin{fulllineitems}
\phantomsection\label{\detokenize{app.base:app.base.models.request_loader}}\pysiglinewithargsret{\sphinxcode{\sphinxupquote{app.base.models.}}\sphinxbfcode{\sphinxupquote{request\_loader}}}{\emph{\DUrole{n}{request}}}{}
\end{fulllineitems}

\index{user\_loader() (dans le module app.base.models)@\spxentry{user\_loader()}\spxextra{dans le module app.base.models}}

\begin{fulllineitems}
\phantomsection\label{\detokenize{app.base:app.base.models.user_loader}}\pysiglinewithargsret{\sphinxcode{\sphinxupquote{app.base.models.}}\sphinxbfcode{\sphinxupquote{user\_loader}}}{\emph{\DUrole{n}{id}}}{}
\end{fulllineitems}



\paragraph{app.base.routes module}
\label{\detokenize{app.base:module-app.base.routes}}\label{\detokenize{app.base:app-base-routes-module}}\index{module@\spxentry{module}!app.base.routes@\spxentry{app.base.routes}}\index{app.base.routes@\spxentry{app.base.routes}!module@\spxentry{module}}
\sphinxAtStartPar
Modified for GRID, 2021

\sphinxAtStartPar
Copyright (c) 2019 \sphinxhyphen{} present AppSeed.us

\sphinxAtStartPar
Gère les routines des connnexions et inscription
\index{access\_forbidden() (dans le module app.base.routes)@\spxentry{access\_forbidden()}\spxextra{dans le module app.base.routes}}

\begin{fulllineitems}
\phantomsection\label{\detokenize{app.base:app.base.routes.access_forbidden}}\pysiglinewithargsret{\sphinxcode{\sphinxupquote{app.base.routes.}}\sphinxbfcode{\sphinxupquote{access\_forbidden}}}{\emph{\DUrole{n}{error}}}{}
\end{fulllineitems}

\index{internal\_error() (dans le module app.base.routes)@\spxentry{internal\_error()}\spxextra{dans le module app.base.routes}}

\begin{fulllineitems}
\phantomsection\label{\detokenize{app.base:app.base.routes.internal_error}}\pysiglinewithargsret{\sphinxcode{\sphinxupquote{app.base.routes.}}\sphinxbfcode{\sphinxupquote{internal\_error}}}{\emph{\DUrole{n}{error}}}{}
\end{fulllineitems}

\index{login() (dans le module app.base.routes)@\spxentry{login()}\spxextra{dans le module app.base.routes}}

\begin{fulllineitems}
\phantomsection\label{\detokenize{app.base:app.base.routes.login}}\pysiglinewithargsret{\sphinxcode{\sphinxupquote{app.base.routes.}}\sphinxbfcode{\sphinxupquote{login}}}{}{}
\end{fulllineitems}

\index{logout() (dans le module app.base.routes)@\spxentry{logout()}\spxextra{dans le module app.base.routes}}

\begin{fulllineitems}
\phantomsection\label{\detokenize{app.base:app.base.routes.logout}}\pysiglinewithargsret{\sphinxcode{\sphinxupquote{app.base.routes.}}\sphinxbfcode{\sphinxupquote{logout}}}{}{}
\end{fulllineitems}

\index{not\_found\_error() (dans le module app.base.routes)@\spxentry{not\_found\_error()}\spxextra{dans le module app.base.routes}}

\begin{fulllineitems}
\phantomsection\label{\detokenize{app.base:app.base.routes.not_found_error}}\pysiglinewithargsret{\sphinxcode{\sphinxupquote{app.base.routes.}}\sphinxbfcode{\sphinxupquote{not\_found\_error}}}{\emph{\DUrole{n}{error}}}{}
\end{fulllineitems}

\index{register() (dans le module app.base.routes)@\spxentry{register()}\spxextra{dans le module app.base.routes}}

\begin{fulllineitems}
\phantomsection\label{\detokenize{app.base:app.base.routes.register}}\pysiglinewithargsret{\sphinxcode{\sphinxupquote{app.base.routes.}}\sphinxbfcode{\sphinxupquote{register}}}{}{}
\end{fulllineitems}

\index{route\_default() (dans le module app.base.routes)@\spxentry{route\_default()}\spxextra{dans le module app.base.routes}}

\begin{fulllineitems}
\phantomsection\label{\detokenize{app.base:app.base.routes.route_default}}\pysiglinewithargsret{\sphinxcode{\sphinxupquote{app.base.routes.}}\sphinxbfcode{\sphinxupquote{route\_default}}}{}{}
\end{fulllineitems}

\index{shutdown() (dans le module app.base.routes)@\spxentry{shutdown()}\spxextra{dans le module app.base.routes}}

\begin{fulllineitems}
\phantomsection\label{\detokenize{app.base:app.base.routes.shutdown}}\pysiglinewithargsret{\sphinxcode{\sphinxupquote{app.base.routes.}}\sphinxbfcode{\sphinxupquote{shutdown}}}{}{}
\end{fulllineitems}

\index{unauthorized\_handler() (dans le module app.base.routes)@\spxentry{unauthorized\_handler()}\spxextra{dans le module app.base.routes}}

\begin{fulllineitems}
\phantomsection\label{\detokenize{app.base:app.base.routes.unauthorized_handler}}\pysiglinewithargsret{\sphinxcode{\sphinxupquote{app.base.routes.}}\sphinxbfcode{\sphinxupquote{unauthorized\_handler}}}{}{}
\end{fulllineitems}



\paragraph{app.base.util module}
\label{\detokenize{app.base:module-app.base.util}}\label{\detokenize{app.base:app-base-util-module}}\index{module@\spxentry{module}!app.base.util@\spxentry{app.base.util}}\index{app.base.util@\spxentry{app.base.util}!module@\spxentry{module}}
\sphinxAtStartPar
Modified for GRID, 2021

\sphinxAtStartPar
Copyright (c) 2019 \sphinxhyphen{} present AppSeed.us
\index{hash\_pass() (dans le module app.base.util)@\spxentry{hash\_pass()}\spxextra{dans le module app.base.util}}

\begin{fulllineitems}
\phantomsection\label{\detokenize{app.base:app.base.util.hash_pass}}\pysiglinewithargsret{\sphinxcode{\sphinxupquote{app.base.util.}}\sphinxbfcode{\sphinxupquote{hash\_pass}}}{\emph{\DUrole{n}{password}}}{}
\sphinxAtStartPar
Hash mot de passe SHA\sphinxhyphen{}256

\end{fulllineitems}

\index{verify\_pass() (dans le module app.base.util)@\spxentry{verify\_pass()}\spxextra{dans le module app.base.util}}

\begin{fulllineitems}
\phantomsection\label{\detokenize{app.base:app.base.util.verify_pass}}\pysiglinewithargsret{\sphinxcode{\sphinxupquote{app.base.util.}}\sphinxbfcode{\sphinxupquote{verify\_pass}}}{\emph{\DUrole{n}{provided\_password}}, \emph{\DUrole{n}{stored\_password}}}{}
\sphinxAtStartPar
Verification du mot de passe par Hash

\end{fulllineitems}



\paragraph{Module contents}
\label{\detokenize{app.base:module-app.base}}\label{\detokenize{app.base:module-contents}}\index{module@\spxentry{module}!app.base@\spxentry{app.base}}\index{app.base@\spxentry{app.base}!module@\spxentry{module}}
\sphinxAtStartPar
Modified for GRID, 2021

\sphinxAtStartPar
Copyright (c) 2019 \sphinxhyphen{} present AppSeed.us


\subsubsection{app.home package}
\label{\detokenize{app.home:app-home-package}}\label{\detokenize{app.home::doc}}

\paragraph{Subpackages}
\label{\detokenize{app.home:subpackages}}

\subparagraph{app.home.content\_gen package}
\label{\detokenize{app.home.content_gen:app-home-content-gen-package}}\label{\detokenize{app.home.content_gen::doc}}

\subparagraph{Submodules}
\label{\detokenize{app.home.content_gen:submodules}}

\subparagraph{app.home.content\_gen.graph\_generation module}
\label{\detokenize{app.home.content_gen:module-app.home.content_gen.graph_generation}}\label{\detokenize{app.home.content_gen:app-home-content-gen-graph-generation-module}}\index{module@\spxentry{module}!app.home.content\_gen.graph\_generation@\spxentry{app.home.content\_gen.graph\_generation}}\index{app.home.content\_gen.graph\_generation@\spxentry{app.home.content\_gen.graph\_generation}!module@\spxentry{module}}
\sphinxAtStartPar
© GRID Team, 2021
\index{BulletChart (classe dans app.home.content\_gen.graph\_generation)@\spxentry{BulletChart}\spxextra{classe dans app.home.content\_gen.graph\_generation}}

\begin{fulllineitems}
\phantomsection\label{\detokenize{app.home.content_gen:app.home.content_gen.graph_generation.BulletChart}}\pysiglinewithargsret{\sphinxbfcode{\sphinxupquote{class }}\sphinxcode{\sphinxupquote{app.home.content\_gen.graph\_generation.}}\sphinxbfcode{\sphinxupquote{BulletChart}}}{\emph{\DUrole{n}{indic}}, \emph{\DUrole{n}{indic\_name}}}{}
\sphinxAtStartPar
Bases : \sphinxcode{\sphinxupquote{object}}

\sphinxAtStartPar
Cette classe génère une échelle à 3 couleurs pour un indicateur donné
\begin{quote}\begin{description}
\item[{Paramètres}] \leavevmode\begin{itemize}
\item {} 
\sphinxAtStartPar
\sphinxstyleliteralstrong{\sphinxupquote{indic}} (\sphinxstyleliteralemphasis{\sphinxupquote{str}}) \textendash{} le code indicateur au format Ex, Sx ou Gx (où x est un int)

\item {} 
\sphinxAtStartPar
\sphinxstyleliteralstrong{\sphinxupquote{indic\_name}} \textendash{} le nom de l’indicateur utilisé pour le titre

\end{itemize}

\end{description}\end{quote}
\index{plot() (méthode app.home.content\_gen.graph\_generation.BulletChart)@\spxentry{plot()}\spxextra{méthode app.home.content\_gen.graph\_generation.BulletChart}}

\begin{fulllineitems}
\phantomsection\label{\detokenize{app.home.content_gen:app.home.content_gen.graph_generation.BulletChart.plot}}\pysiglinewithargsret{\sphinxbfcode{\sphinxupquote{plot}}}{}{}
\sphinxAtStartPar
Les données sont importées depuis l’\_\_init\_\_
\begin{quote}\begin{description}
\item[{Renvoie}] \leavevmode
\sphinxAtStartPar
objet json contenant le plot

\item[{Type renvoyé}] \leavevmode
\sphinxAtStartPar
json

\end{description}\end{quote}

\end{fulllineitems}


\end{fulllineitems}

\index{CaniculePlot (classe dans app.home.content\_gen.graph\_generation)@\spxentry{CaniculePlot}\spxextra{classe dans app.home.content\_gen.graph\_generation}}

\begin{fulllineitems}
\phantomsection\label{\detokenize{app.home.content_gen:app.home.content_gen.graph_generation.CaniculePlot}}\pysigline{\sphinxbfcode{\sphinxupquote{class }}\sphinxcode{\sphinxupquote{app.home.content\_gen.graph\_generation.}}\sphinxbfcode{\sphinxupquote{CaniculePlot}}}
\sphinxAtStartPar
Bases : \sphinxcode{\sphinxupquote{object}}

\sphinxAtStartPar
Cette classe génère le graphique des canicules dans la page Environnement.
Les données sont importées directement
\index{find\_closest() (méthode app.home.content\_gen.graph\_generation.CaniculePlot)@\spxentry{find\_closest()}\spxextra{méthode app.home.content\_gen.graph\_generation.CaniculePlot}}

\begin{fulllineitems}
\phantomsection\label{\detokenize{app.home.content_gen:app.home.content_gen.graph_generation.CaniculePlot.find_closest}}\pysiglinewithargsret{\sphinxbfcode{\sphinxupquote{find\_closest}}}{}{}
\sphinxAtStartPar
Sur la base de la localisation de la PME, recherche le point de donnée le plus proche.
Ces données deviennent les variables self.lat et self.lon

\end{fulllineitems}

\index{main() (méthode app.home.content\_gen.graph\_generation.CaniculePlot)@\spxentry{main()}\spxextra{méthode app.home.content\_gen.graph\_generation.CaniculePlot}}

\begin{fulllineitems}
\phantomsection\label{\detokenize{app.home.content_gen:app.home.content_gen.graph_generation.CaniculePlot.main}}\pysiglinewithargsret{\sphinxbfcode{\sphinxupquote{main}}}{}{}
\sphinxAtStartPar
Fonction principale de la classe
\begin{quote}\begin{description}
\item[{Renvoie}] \leavevmode
\sphinxAtStartPar
objet json

\item[{Type renvoyé}] \leavevmode
\sphinxAtStartPar
json

\end{description}\end{quote}

\end{fulllineitems}

\index{plot() (méthode app.home.content\_gen.graph\_generation.CaniculePlot)@\spxentry{plot()}\spxextra{méthode app.home.content\_gen.graph\_generation.CaniculePlot}}

\begin{fulllineitems}
\phantomsection\label{\detokenize{app.home.content_gen:app.home.content_gen.graph_generation.CaniculePlot.plot}}\pysiglinewithargsret{\sphinxbfcode{\sphinxupquote{plot}}}{}{}
\sphinxAtStartPar
Plot un graphique ligne et stocke l’object json dans self.graphjson

\end{fulllineitems}


\end{fulllineitems}

\index{FinancialChart (classe dans app.home.content\_gen.graph\_generation)@\spxentry{FinancialChart}\spxextra{classe dans app.home.content\_gen.graph\_generation}}

\begin{fulllineitems}
\phantomsection\label{\detokenize{app.home.content_gen:app.home.content_gen.graph_generation.FinancialChart}}\pysiglinewithargsret{\sphinxbfcode{\sphinxupquote{class }}\sphinxcode{\sphinxupquote{app.home.content\_gen.graph\_generation.}}\sphinxbfcode{\sphinxupquote{FinancialChart}}}{\emph{\DUrole{o}{*}\DUrole{n}{args}}}{}
\sphinxAtStartPar
Bases : \sphinxcode{\sphinxupquote{object}}

\sphinxAtStartPar
Cette classe génère les diagrammes pour la partie finance
\begin{quote}\begin{description}
\item[{Paramètres}] \leavevmode
\sphinxAtStartPar
\sphinxstyleliteralstrong{\sphinxupquote{**args}} \textendash{} 
\sphinxAtStartPar
le code indicateur au format Ex, Sx ou Gx (où x est un int)


\end{description}\end{quote}
\index{plot\_bar() (méthode app.home.content\_gen.graph\_generation.FinancialChart)@\spxentry{plot\_bar()}\spxextra{méthode app.home.content\_gen.graph\_generation.FinancialChart}}

\begin{fulllineitems}
\phantomsection\label{\detokenize{app.home.content_gen:app.home.content_gen.graph_generation.FinancialChart.plot_bar}}\pysiglinewithargsret{\sphinxbfcode{\sphinxupquote{plot\_bar}}}{}{}
\sphinxAtStartPar
Les données sont importées depuis l’\_\_init\_\_. Génère un graphique barre
\begin{quote}\begin{description}
\item[{Renvoie}] \leavevmode
\sphinxAtStartPar
list d’objet json

\item[{Type renvoyé}] \leavevmode
\sphinxAtStartPar
list{[}json{]}

\end{description}\end{quote}

\end{fulllineitems}

\index{plot\_mltpl\_line() (méthode app.home.content\_gen.graph\_generation.FinancialChart)@\spxentry{plot\_mltpl\_line()}\spxextra{méthode app.home.content\_gen.graph\_generation.FinancialChart}}

\begin{fulllineitems}
\phantomsection\label{\detokenize{app.home.content_gen:app.home.content_gen.graph_generation.FinancialChart.plot_mltpl_line}}\pysiglinewithargsret{\sphinxbfcode{\sphinxupquote{plot\_mltpl\_line}}}{}{}
\sphinxAtStartPar
Les données sont importées depuis l’\_\_init\_\_. Génère un graphique ligne avec 2 axes y
\begin{quote}\begin{description}
\item[{Renvoie}] \leavevmode
\sphinxAtStartPar
list d’objet json

\item[{Type renvoyé}] \leavevmode
\sphinxAtStartPar
list{[}json{]}

\end{description}\end{quote}

\end{fulllineitems}

\index{plot\_sgl\_line() (méthode app.home.content\_gen.graph\_generation.FinancialChart)@\spxentry{plot\_sgl\_line()}\spxextra{méthode app.home.content\_gen.graph\_generation.FinancialChart}}

\begin{fulllineitems}
\phantomsection\label{\detokenize{app.home.content_gen:app.home.content_gen.graph_generation.FinancialChart.plot_sgl_line}}\pysiglinewithargsret{\sphinxbfcode{\sphinxupquote{plot\_sgl\_line}}}{}{}
\sphinxAtStartPar
Les données sont importées depuis l’\_\_init\_\_. Génère un graphique ligne
\begin{quote}\begin{description}
\item[{Renvoie}] \leavevmode
\sphinxAtStartPar
list d’objet json

\item[{Type renvoyé}] \leavevmode
\sphinxAtStartPar
list{[}json{]}

\end{description}\end{quote}

\end{fulllineitems}


\end{fulllineitems}

\index{PieChart (classe dans app.home.content\_gen.graph\_generation)@\spxentry{PieChart}\spxextra{classe dans app.home.content\_gen.graph\_generation}}

\begin{fulllineitems}
\phantomsection\label{\detokenize{app.home.content_gen:app.home.content_gen.graph_generation.PieChart}}\pysiglinewithargsret{\sphinxbfcode{\sphinxupquote{class }}\sphinxcode{\sphinxupquote{app.home.content\_gen.graph\_generation.}}\sphinxbfcode{\sphinxupquote{PieChart}}}{\emph{\DUrole{n}{indic}}, \emph{\DUrole{n}{indic\_name}}}{}
\sphinxAtStartPar
Bases : \sphinxcode{\sphinxupquote{object}}

\sphinxAtStartPar
Cette classe génère les diagrames camembert
\begin{quote}\begin{description}
\item[{Paramètres}] \leavevmode\begin{itemize}
\item {} 
\sphinxAtStartPar
\sphinxstyleliteralstrong{\sphinxupquote{indic}} (\sphinxstyleliteralemphasis{\sphinxupquote{str}}) \textendash{} le code indicateur au format Ex, Sx ou Gx (où x est un int)

\item {} 
\sphinxAtStartPar
\sphinxstyleliteralstrong{\sphinxupquote{indic\_name}} \textendash{} le nom de l’indicateur utiliser pour le titre

\end{itemize}

\end{description}\end{quote}
\index{plot() (méthode app.home.content\_gen.graph\_generation.PieChart)@\spxentry{plot()}\spxextra{méthode app.home.content\_gen.graph\_generation.PieChart}}

\begin{fulllineitems}
\phantomsection\label{\detokenize{app.home.content_gen:app.home.content_gen.graph_generation.PieChart.plot}}\pysiglinewithargsret{\sphinxbfcode{\sphinxupquote{plot}}}{}{}
\sphinxAtStartPar
Les données sont importées depuis l’\_\_init\_\_
\begin{quote}\begin{description}
\item[{Renvoie}] \leavevmode
\sphinxAtStartPar
objet json contenant le plot

\item[{Type renvoyé}] \leavevmode
\sphinxAtStartPar
json

\end{description}\end{quote}

\end{fulllineitems}


\end{fulllineitems}



\subparagraph{app.home.content\_gen.index\_renderer module}
\label{\detokenize{app.home.content_gen:module-app.home.content_gen.index_renderer}}\label{\detokenize{app.home.content_gen:app-home-content-gen-index-renderer-module}}\index{module@\spxentry{module}!app.home.content\_gen.index\_renderer@\spxentry{app.home.content\_gen.index\_renderer}}\index{app.home.content\_gen.index\_renderer@\spxentry{app.home.content\_gen.index\_renderer}!module@\spxentry{module}}
\sphinxAtStartPar
© GRID Team, 2021
\index{CriticalAlert (classe dans app.home.content\_gen.index\_renderer)@\spxentry{CriticalAlert}\spxextra{classe dans app.home.content\_gen.index\_renderer}}

\begin{fulllineitems}
\phantomsection\label{\detokenize{app.home.content_gen:app.home.content_gen.index_renderer.CriticalAlert}}\pysigline{\sphinxbfcode{\sphinxupquote{class }}\sphinxcode{\sphinxupquote{app.home.content\_gen.index\_renderer.}}\sphinxbfcode{\sphinxupquote{CriticalAlert}}}
\sphinxAtStartPar
Bases : \sphinxcode{\sphinxupquote{object}}

\sphinxAtStartPar
Cette classe donne las liste des indicateurs considérés comme critique.
\index{main() (méthode app.home.content\_gen.index\_renderer.CriticalAlert)@\spxentry{main()}\spxextra{méthode app.home.content\_gen.index\_renderer.CriticalAlert}}

\begin{fulllineitems}
\phantomsection\label{\detokenize{app.home.content_gen:app.home.content_gen.index_renderer.CriticalAlert.main}}\pysiglinewithargsret{\sphinxbfcode{\sphinxupquote{main}}}{}{}~\begin{quote}\begin{description}
\item[{Renvoie}] \leavevmode
\sphinxAtStartPar
liste de listes (une par indicateur) contenant pour chaque la liste des indicateurs critiques

\item[{Type renvoyé}] \leavevmode
\sphinxAtStartPar
list

\end{description}\end{quote}

\end{fulllineitems}


\end{fulllineitems}

\index{Scoring (classe dans app.home.content\_gen.index\_renderer)@\spxentry{Scoring}\spxextra{classe dans app.home.content\_gen.index\_renderer}}

\begin{fulllineitems}
\phantomsection\label{\detokenize{app.home.content_gen:app.home.content_gen.index_renderer.Scoring}}\pysigline{\sphinxbfcode{\sphinxupquote{class }}\sphinxcode{\sphinxupquote{app.home.content\_gen.index\_renderer.}}\sphinxbfcode{\sphinxupquote{Scoring}}}
\sphinxAtStartPar
Bases : \sphinxcode{\sphinxupquote{object}}

\sphinxAtStartPar
Cette classe donne les données nécessaires au rendu des gauges indiquant les scores ESG
\index{bin() (méthode app.home.content\_gen.index\_renderer.Scoring)@\spxentry{bin()}\spxextra{méthode app.home.content\_gen.index\_renderer.Scoring}}

\begin{fulllineitems}
\phantomsection\label{\detokenize{app.home.content_gen:app.home.content_gen.index_renderer.Scoring.bin}}\pysiglinewithargsret{\sphinxbfcode{\sphinxupquote{bin}}}{}{}
\sphinxAtStartPar
Génère les intervalles autour de la valeur moyenne

\end{fulllineitems}

\index{main() (méthode app.home.content\_gen.index\_renderer.Scoring)@\spxentry{main()}\spxextra{méthode app.home.content\_gen.index\_renderer.Scoring}}

\begin{fulllineitems}
\phantomsection\label{\detokenize{app.home.content_gen:app.home.content_gen.index_renderer.Scoring.main}}\pysiglinewithargsret{\sphinxbfcode{\sphinxupquote{main}}}{}{}~\begin{quote}\begin{description}
\item[{Renvoie}] \leavevmode
\sphinxAtStartPar
liste de listes (une par indicateur) contenant pour chaque: sa valeur, la valeur max de l’echelle, une liste avec les intervalles de couleurs

\item[{Type renvoyé}] \leavevmode
\sphinxAtStartPar
list

\end{description}\end{quote}

\end{fulllineitems}


\end{fulllineitems}



\subparagraph{app.home.content\_gen.map\_generation module}
\label{\detokenize{app.home.content_gen:module-app.home.content_gen.map_generation}}\label{\detokenize{app.home.content_gen:app-home-content-gen-map-generation-module}}\index{module@\spxentry{module}!app.home.content\_gen.map\_generation@\spxentry{app.home.content\_gen.map\_generation}}\index{app.home.content\_gen.map\_generation@\spxentry{app.home.content\_gen.map\_generation}!module@\spxentry{module}}
\sphinxAtStartPar
© GRID Team, 2021
\index{CaniculePlot (classe dans app.home.content\_gen.map\_generation)@\spxentry{CaniculePlot}\spxextra{classe dans app.home.content\_gen.map\_generation}}

\begin{fulllineitems}
\phantomsection\label{\detokenize{app.home.content_gen:app.home.content_gen.map_generation.CaniculePlot}}\pysigline{\sphinxbfcode{\sphinxupquote{class }}\sphinxcode{\sphinxupquote{app.home.content\_gen.map\_generation.}}\sphinxbfcode{\sphinxupquote{CaniculePlot}}}
\sphinxAtStartPar
Bases : \sphinxcode{\sphinxupquote{object}}

\sphinxAtStartPar
Cette classe génère une heat map des canicules sur la base des données de Copernicus.

\sphinxAtStartPar
Les données ont été pré\sphinxhyphen{}traitées et stockées dans le même répertoire.
\index{main() (méthode app.home.content\_gen.map\_generation.CaniculePlot)@\spxentry{main()}\spxextra{méthode app.home.content\_gen.map\_generation.CaniculePlot}}

\begin{fulllineitems}
\phantomsection\label{\detokenize{app.home.content_gen:app.home.content_gen.map_generation.CaniculePlot.main}}\pysiglinewithargsret{\sphinxbfcode{\sphinxupquote{main}}}{}{}
\sphinxAtStartPar
Fonction lançant le tout
\begin{quote}\begin{description}
\item[{Renvoie}] \leavevmode
\sphinxAtStartPar
objet json

\item[{Type renvoyé}] \leavevmode
\sphinxAtStartPar
json

\end{description}\end{quote}

\end{fulllineitems}

\index{plot\_at\_date() (méthode app.home.content\_gen.map\_generation.CaniculePlot)@\spxentry{plot\_at\_date()}\spxextra{méthode app.home.content\_gen.map\_generation.CaniculePlot}}

\begin{fulllineitems}
\phantomsection\label{\detokenize{app.home.content_gen:app.home.content_gen.map_generation.CaniculePlot.plot_at_date}}\pysiglinewithargsret{\sphinxbfcode{\sphinxupquote{plot\_at\_date}}}{}{}
\sphinxAtStartPar
Crée un carte pour un date données
\begin{quote}\begin{description}
\item[{Renvoie}] \leavevmode
\sphinxAtStartPar
objet json

\item[{Type renvoyé}] \leavevmode
\sphinxAtStartPar
json

\end{description}\end{quote}

\end{fulllineitems}

\index{plot\_cursor() (méthode app.home.content\_gen.map\_generation.CaniculePlot)@\spxentry{plot\_cursor()}\spxextra{méthode app.home.content\_gen.map\_generation.CaniculePlot}}

\begin{fulllineitems}
\phantomsection\label{\detokenize{app.home.content_gen:app.home.content_gen.map_generation.CaniculePlot.plot_cursor}}\pysiglinewithargsret{\sphinxbfcode{\sphinxupquote{plot\_cursor}}}{}{}
\sphinxAtStartPar
Crée un carte pour différentes dates avec un slider temporel
(dates définies dans la variable \sphinxtitleref{list\_date})
\begin{quote}\begin{description}
\item[{Renvoie}] \leavevmode
\sphinxAtStartPar
objet json

\item[{Type renvoyé}] \leavevmode
\sphinxAtStartPar
json

\end{description}\end{quote}

\end{fulllineitems}

\index{read\_json() (méthode app.home.content\_gen.map\_generation.CaniculePlot)@\spxentry{read\_json()}\spxextra{méthode app.home.content\_gen.map\_generation.CaniculePlot}}

\begin{fulllineitems}
\phantomsection\label{\detokenize{app.home.content_gen:app.home.content_gen.map_generation.CaniculePlot.read_json}}\pysiglinewithargsret{\sphinxbfcode{\sphinxupquote{read\_json}}}{}{}
\end{fulllineitems}


\end{fulllineitems}

\index{FirePlot (classe dans app.home.content\_gen.map\_generation)@\spxentry{FirePlot}\spxextra{classe dans app.home.content\_gen.map\_generation}}

\begin{fulllineitems}
\phantomsection\label{\detokenize{app.home.content_gen:app.home.content_gen.map_generation.FirePlot}}\pysigline{\sphinxbfcode{\sphinxupquote{class }}\sphinxcode{\sphinxupquote{app.home.content\_gen.map\_generation.}}\sphinxbfcode{\sphinxupquote{FirePlot}}}
\sphinxAtStartPar
Bases : \sphinxcode{\sphinxupquote{object}}

\sphinxAtStartPar
Cette classe génère une carte avec un scatter plot des risques incendies sur la base des données de Copernicus.

\sphinxAtStartPar
Les données ont été pré\sphinxhyphen{}traitées et stockées dans le même répertoire.
\index{color\_scale() (méthode app.home.content\_gen.map\_generation.FirePlot)@\spxentry{color\_scale()}\spxextra{méthode app.home.content\_gen.map\_generation.FirePlot}}

\begin{fulllineitems}
\phantomsection\label{\detokenize{app.home.content_gen:app.home.content_gen.map_generation.FirePlot.color_scale}}\pysiglinewithargsret{\sphinxbfcode{\sphinxupquote{color\_scale}}}{\emph{\DUrole{n}{zmax}}}{}
\sphinxAtStartPar
Cette fonction accomplit 2 choses en parallèle: création d’une echelle de couleurs pour correpondre au Fire Index européen et
trouve les valeurs centrales de chacun des intervalles utilisés pour afficher l’echelle de couleur annotée
\begin{quote}\begin{description}
\item[{Renvoie}] \leavevmode
\sphinxAtStartPar
liste de l’echelle de couleurs normée (i.e. valeurs entre 0 et 1) et liste du centre des intervalles

\item[{Type renvoyé}] \leavevmode
\sphinxAtStartPar
list

\end{description}\end{quote}

\end{fulllineitems}

\index{main() (méthode app.home.content\_gen.map\_generation.FirePlot)@\spxentry{main()}\spxextra{méthode app.home.content\_gen.map\_generation.FirePlot}}

\begin{fulllineitems}
\phantomsection\label{\detokenize{app.home.content_gen:app.home.content_gen.map_generation.FirePlot.main}}\pysiglinewithargsret{\sphinxbfcode{\sphinxupquote{main}}}{}{}
\sphinxAtStartPar
Fonction lançant le tout
\begin{quote}\begin{description}
\item[{Renvoie}] \leavevmode
\sphinxAtStartPar
objet json

\item[{Type renvoyé}] \leavevmode
\sphinxAtStartPar
json

\end{description}\end{quote}

\end{fulllineitems}

\index{plot\_at\_date() (méthode app.home.content\_gen.map\_generation.FirePlot)@\spxentry{plot\_at\_date()}\spxextra{méthode app.home.content\_gen.map\_generation.FirePlot}}

\begin{fulllineitems}
\phantomsection\label{\detokenize{app.home.content_gen:app.home.content_gen.map_generation.FirePlot.plot_at_date}}\pysiglinewithargsret{\sphinxbfcode{\sphinxupquote{plot\_at\_date}}}{}{}
\sphinxAtStartPar
Crée un carte pour une date donnée
\begin{quote}\begin{description}
\item[{Renvoie}] \leavevmode
\sphinxAtStartPar
objet json

\item[{Type renvoyé}] \leavevmode
\sphinxAtStartPar
json

\end{description}\end{quote}

\end{fulllineitems}

\index{plot\_cursor() (méthode app.home.content\_gen.map\_generation.FirePlot)@\spxentry{plot\_cursor()}\spxextra{méthode app.home.content\_gen.map\_generation.FirePlot}}

\begin{fulllineitems}
\phantomsection\label{\detokenize{app.home.content_gen:app.home.content_gen.map_generation.FirePlot.plot_cursor}}\pysiglinewithargsret{\sphinxbfcode{\sphinxupquote{plot\_cursor}}}{}{}
\sphinxAtStartPar
Crée un carte pour différentes dates avec un slider temporel
(dates définies dans la variable \sphinxtitleref{list\_date})
\begin{quote}\begin{description}
\item[{Renvoie}] \leavevmode
\sphinxAtStartPar
objet json

\item[{Type renvoyé}] \leavevmode
\sphinxAtStartPar
json

\end{description}\end{quote}

\end{fulllineitems}

\index{read\_json() (méthode app.home.content\_gen.map\_generation.FirePlot)@\spxentry{read\_json()}\spxextra{méthode app.home.content\_gen.map\_generation.FirePlot}}

\begin{fulllineitems}
\phantomsection\label{\detokenize{app.home.content_gen:app.home.content_gen.map_generation.FirePlot.read_json}}\pysiglinewithargsret{\sphinxbfcode{\sphinxupquote{read\_json}}}{}{}
\sphinxAtStartPar
Lecture du fichier .json et tri de l’index

\end{fulllineitems}


\end{fulllineitems}



\subparagraph{app.home.content\_gen.questionaire module}
\label{\detokenize{app.home.content_gen:module-app.home.content_gen.questionaire}}\label{\detokenize{app.home.content_gen:app-home-content-gen-questionaire-module}}\index{module@\spxentry{module}!app.home.content\_gen.questionaire@\spxentry{app.home.content\_gen.questionaire}}\index{app.home.content\_gen.questionaire@\spxentry{app.home.content\_gen.questionaire}!module@\spxentry{module}}
\sphinxAtStartPar
© GRID Team, 2021
\index{QuestionairesAgri (classe dans app.home.content\_gen.questionaire)@\spxentry{QuestionairesAgri}\spxextra{classe dans app.home.content\_gen.questionaire}}

\begin{fulllineitems}
\phantomsection\label{\detokenize{app.home.content_gen:app.home.content_gen.questionaire.QuestionairesAgri}}\pysiglinewithargsret{\sphinxbfcode{\sphinxupquote{class }}\sphinxcode{\sphinxupquote{app.home.content\_gen.questionaire.}}\sphinxbfcode{\sphinxupquote{QuestionairesAgri}}}{\emph{\DUrole{o}{*}\DUrole{n}{args}}, \emph{\DUrole{o}{**}\DUrole{n}{kwargs}}}{}
\sphinxAtStartPar
Bases : \sphinxcode{\sphinxupquote{flask\_wtf.form.FlaskForm}}

\sphinxAtStartPar
Cette classe génère le questionaire Flask nécessaire au rendu HTML
\index{address (attribut app.home.content\_gen.questionaire.QuestionairesAgri)@\spxentry{address}\spxextra{attribut app.home.content\_gen.questionaire.QuestionairesAgri}}

\begin{fulllineitems}
\phantomsection\label{\detokenize{app.home.content_gen:app.home.content_gen.questionaire.QuestionairesAgri.address}}\pysigline{\sphinxbfcode{\sphinxupquote{address}}\sphinxbfcode{\sphinxupquote{ = \textless{}UnboundField(TextField, (\textquotesingle{}Address\textquotesingle{},), \{\})\textgreater{}}}}
\end{fulllineitems}

\index{age (attribut app.home.content\_gen.questionaire.QuestionairesAgri)@\spxentry{age}\spxextra{attribut app.home.content\_gen.questionaire.QuestionairesAgri}}

\begin{fulllineitems}
\phantomsection\label{\detokenize{app.home.content_gen:app.home.content_gen.questionaire.QuestionairesAgri.age}}\pysigline{\sphinxbfcode{\sphinxupquote{age}}\sphinxbfcode{\sphinxupquote{ = \textless{}UnboundField(TextField, (\textquotesingle{}Age\textquotesingle{},), \{\})\textgreater{}}}}
\end{fulllineitems}

\index{autract (attribut app.home.content\_gen.questionaire.QuestionairesAgri)@\spxentry{autract}\spxextra{attribut app.home.content\_gen.questionaire.QuestionairesAgri}}

\begin{fulllineitems}
\phantomsection\label{\detokenize{app.home.content_gen:app.home.content_gen.questionaire.QuestionairesAgri.autract}}\pysigline{\sphinxbfcode{\sphinxupquote{autract}}\sphinxbfcode{\sphinxupquote{ = \textless{}UnboundField(TextField, (\textquotesingle{}autre activite\textquotesingle{},), \{\})\textgreater{}}}}
\end{fulllineitems}

\index{autrcult (attribut app.home.content\_gen.questionaire.QuestionairesAgri)@\spxentry{autrcult}\spxextra{attribut app.home.content\_gen.questionaire.QuestionairesAgri}}

\begin{fulllineitems}
\phantomsection\label{\detokenize{app.home.content_gen:app.home.content_gen.questionaire.QuestionairesAgri.autrcult}}\pysigline{\sphinxbfcode{\sphinxupquote{autrcult}}\sphinxbfcode{\sphinxupquote{ = \textless{}UnboundField(SelectField, (\textquotesingle{}autre cultures\textquotesingle{},), \{\textquotesingle{}choices\textquotesingle{}: {[}(\textquotesingle{}init\textquotesingle{}, \textquotesingle{}sélectionnez la proposition\textquotesingle{}), (\textquotesingle{}y\textquotesingle{}, \textquotesingle{}oui\textquotesingle{}), (\textquotesingle{}n\textquotesingle{}, \textquotesingle{}non\textquotesingle{}){]}\})\textgreater{}}}}
\end{fulllineitems}

\index{autrecertif (attribut app.home.content\_gen.questionaire.QuestionairesAgri)@\spxentry{autrecertif}\spxextra{attribut app.home.content\_gen.questionaire.QuestionairesAgri}}

\begin{fulllineitems}
\phantomsection\label{\detokenize{app.home.content_gen:app.home.content_gen.questionaire.QuestionairesAgri.autrecertif}}\pysigline{\sphinxbfcode{\sphinxupquote{autrecertif}}\sphinxbfcode{\sphinxupquote{ = \textless{}UnboundField(TextField, (\textquotesingle{}autre certication\textquotesingle{},), \{\})\textgreater{}}}}
\end{fulllineitems}

\index{autrequal (attribut app.home.content\_gen.questionaire.QuestionairesAgri)@\spxentry{autrequal}\spxextra{attribut app.home.content\_gen.questionaire.QuestionairesAgri}}

\begin{fulllineitems}
\phantomsection\label{\detokenize{app.home.content_gen:app.home.content_gen.questionaire.QuestionairesAgri.autrequal}}\pysigline{\sphinxbfcode{\sphinxupquote{autrequal}}\sphinxbfcode{\sphinxupquote{ = \textless{}UnboundField(TextField, (\textquotesingle{}autre qualite\textquotesingle{},), \{\})\textgreater{}}}}
\end{fulllineitems}

\index{cepage (attribut app.home.content\_gen.questionaire.QuestionairesAgri)@\spxentry{cepage}\spxextra{attribut app.home.content\_gen.questionaire.QuestionairesAgri}}

\begin{fulllineitems}
\phantomsection\label{\detokenize{app.home.content_gen:app.home.content_gen.questionaire.QuestionairesAgri.cepage}}\pysigline{\sphinxbfcode{\sphinxupquote{cepage}}\sphinxbfcode{\sphinxupquote{ = \textless{}UnboundField(SelectMultipleField, (), \{\textquotesingle{}choices\textquotesingle{}: {[}(\textquotesingle{}init\textquotesingle{}, \textquotesingle{}sélectionnez la proposition\textquotesingle{}), (\textquotesingle{}cep1\textquotesingle{}, \textquotesingle{}cabernet sauvignon\textquotesingle{}), (\textquotesingle{}cep2\textquotesingle{}, \textquotesingle{}carignan\textquotesingle{}), (\textquotesingle{}cep3\textquotesingle{}, \textquotesingle{}grenache noir\textquotesingle{}), (\textquotesingle{}cep4\textquotesingle{}, \textquotesingle{}syrah\textquotesingle{}), (\textquotesingle{}cep5\textquotesingle{}, \textquotesingle{}muscat\textquotesingle{}), (\textquotesingle{}cep6\textquotesingle{}, \textquotesingle{}chardonnay\textquotesingle{}), (\textquotesingle{}cep7\textquotesingle{}, \textquotesingle{}cinsault\textquotesingle{}){]}\})\textgreater{}}}}
\end{fulllineitems}

\index{certif (attribut app.home.content\_gen.questionaire.QuestionairesAgri)@\spxentry{certif}\spxextra{attribut app.home.content\_gen.questionaire.QuestionairesAgri}}

\begin{fulllineitems}
\phantomsection\label{\detokenize{app.home.content_gen:app.home.content_gen.questionaire.QuestionairesAgri.certif}}\pysigline{\sphinxbfcode{\sphinxupquote{certif}}\sphinxbfcode{\sphinxupquote{ = \textless{}UnboundField(SelectField, (\textquotesingle{}certification\textquotesingle{},), \{\textquotesingle{}choices\textquotesingle{}: {[}(\textquotesingle{}bio\textquotesingle{}, \textquotesingle{}label BIO\textquotesingle{}), (\textquotesingle{}hve\textquotesingle{}, \textquotesingle{}label HVE\textquotesingle{}), (\textquotesingle{}els\textquotesingle{}, \textquotesingle{}autre\textquotesingle{}), (\textquotesingle{}n\textquotesingle{}, \textquotesingle{}aucune\textquotesingle{}){]}\})\textgreater{}}}}
\end{fulllineitems}

\index{etp (attribut app.home.content\_gen.questionaire.QuestionairesAgri)@\spxentry{etp}\spxextra{attribut app.home.content\_gen.questionaire.QuestionairesAgri}}

\begin{fulllineitems}
\phantomsection\label{\detokenize{app.home.content_gen:app.home.content_gen.questionaire.QuestionairesAgri.etp}}\pysigline{\sphinxbfcode{\sphinxupquote{etp}}\sphinxbfcode{\sphinxupquote{ = \textless{}UnboundField(TextField, (\textquotesingle{}etp\textquotesingle{},), \{\})\textgreater{}}}}
\end{fulllineitems}

\index{haie (attribut app.home.content\_gen.questionaire.QuestionairesAgri)@\spxentry{haie}\spxextra{attribut app.home.content\_gen.questionaire.QuestionairesAgri}}

\begin{fulllineitems}
\phantomsection\label{\detokenize{app.home.content_gen:app.home.content_gen.questionaire.QuestionairesAgri.haie}}\pysigline{\sphinxbfcode{\sphinxupquote{haie}}\sphinxbfcode{\sphinxupquote{ = \textless{}UnboundField(SelectField, (\textquotesingle{}Presence haies\textquotesingle{},), \{\textquotesingle{}choices\textquotesingle{}: {[}(\textquotesingle{}init\textquotesingle{}, \textquotesingle{}sélectionnez la proposition\textquotesingle{}), (\textquotesingle{}y1\textquotesingle{}, \textquotesingle{}oui sur toutes les parcelles\textquotesingle{}), (\textquotesingle{}y2\textquotesingle{}, \textquotesingle{}oui sur une partie des parcelles\textquotesingle{}), (\textquotesingle{}no\textquotesingle{}, \textquotesingle{}non\textquotesingle{}){]}\})\textgreater{}}}}
\end{fulllineitems}

\index{ift (attribut app.home.content\_gen.questionaire.QuestionairesAgri)@\spxentry{ift}\spxextra{attribut app.home.content\_gen.questionaire.QuestionairesAgri}}

\begin{fulllineitems}
\phantomsection\label{\detokenize{app.home.content_gen:app.home.content_gen.questionaire.QuestionairesAgri.ift}}\pysigline{\sphinxbfcode{\sphinxupquote{ift}}\sphinxbfcode{\sphinxupquote{ = \textless{}UnboundField(TextField, (\textquotesingle{}ift\textquotesingle{},), \{\})\textgreater{}}}}
\end{fulllineitems}

\index{intrant (attribut app.home.content\_gen.questionaire.QuestionairesAgri)@\spxentry{intrant}\spxextra{attribut app.home.content\_gen.questionaire.QuestionairesAgri}}

\begin{fulllineitems}
\phantomsection\label{\detokenize{app.home.content_gen:app.home.content_gen.questionaire.QuestionairesAgri.intrant}}\pysigline{\sphinxbfcode{\sphinxupquote{intrant}}\sphinxbfcode{\sphinxupquote{ = \textless{}UnboundField(TextField, (\textquotesingle{}intrant\textquotesingle{},), \{\})\textgreater{}}}}
\end{fulllineitems}

\index{irrig (attribut app.home.content\_gen.questionaire.QuestionairesAgri)@\spxentry{irrig}\spxextra{attribut app.home.content\_gen.questionaire.QuestionairesAgri}}

\begin{fulllineitems}
\phantomsection\label{\detokenize{app.home.content_gen:app.home.content_gen.questionaire.QuestionairesAgri.irrig}}\pysigline{\sphinxbfcode{\sphinxupquote{irrig}}\sphinxbfcode{\sphinxupquote{ = \textless{}UnboundField(RadioField, (), \{\textquotesingle{}choices\textquotesingle{}: {[}(\textquotesingle{}init\textquotesingle{}, \textquotesingle{}sélectionnez la proposition\textquotesingle{}), (\textquotesingle{}no\_irrig\textquotesingle{}, \textquotesingle{}aucune irrigation\textquotesingle{}), (\textquotesingle{}yes\_irrig1\textquotesingle{}, \textquotesingle{}oui sur la majorité des parcelles\textquotesingle{}), (\textquotesingle{}yes\_irrig2\textquotesingle{}, \textquotesingle{}oui sur certaines parcelles\textquotesingle{}){]}\})\textgreater{}}}}
\end{fulllineitems}

\index{mutu (attribut app.home.content\_gen.questionaire.QuestionairesAgri)@\spxentry{mutu}\spxextra{attribut app.home.content\_gen.questionaire.QuestionairesAgri}}

\begin{fulllineitems}
\phantomsection\label{\detokenize{app.home.content_gen:app.home.content_gen.questionaire.QuestionairesAgri.mutu}}\pysigline{\sphinxbfcode{\sphinxupquote{mutu}}\sphinxbfcode{\sphinxupquote{ = \textless{}UnboundField(SelectMultipleField, (), \{\textquotesingle{}choices\textquotesingle{}: {[}(\textquotesingle{}init\textquotesingle{}, \textquotesingle{}sélectionnez la proposition\textquotesingle{}), (\textquotesingle{}yes\_mutu\textquotesingle{}, \textquotesingle{}oui\textquotesingle{}), (\textquotesingle{}no\_mutu\textquotesingle{}, \textquotesingle{}non\textquotesingle{}){]}\})\textgreater{}}}}
\end{fulllineitems}

\index{name\_exploit (attribut app.home.content\_gen.questionaire.QuestionairesAgri)@\spxentry{name\_exploit}\spxextra{attribut app.home.content\_gen.questionaire.QuestionairesAgri}}

\begin{fulllineitems}
\phantomsection\label{\detokenize{app.home.content_gen:app.home.content_gen.questionaire.QuestionairesAgri.name_exploit}}\pysigline{\sphinxbfcode{\sphinxupquote{name\_exploit}}\sphinxbfcode{\sphinxupquote{ = \textless{}UnboundField(TextField, (\textquotesingle{}Nom exploitation\textquotesingle{},), \{\})\textgreater{}}}}
\end{fulllineitems}

\index{qual (attribut app.home.content\_gen.questionaire.QuestionairesAgri)@\spxentry{qual}\spxextra{attribut app.home.content\_gen.questionaire.QuestionairesAgri}}

\begin{fulllineitems}
\phantomsection\label{\detokenize{app.home.content_gen:app.home.content_gen.questionaire.QuestionairesAgri.qual}}\pysigline{\sphinxbfcode{\sphinxupquote{qual}}\sphinxbfcode{\sphinxupquote{ = \textless{}UnboundField(SelectField, (\textquotesingle{}certification qualite\textquotesingle{},), \{\textquotesingle{}choices\textquotesingle{}: {[}(\textquotesingle{}init\textquotesingle{}, \textquotesingle{}sélectionnez la proposition\textquotesingle{}), (\textquotesingle{}igp\textquotesingle{}, \textquotesingle{}IGP\textquotesingle{}), (\textquotesingle{}aop\textquotesingle{}, \textquotesingle{}AOP\textquotesingle{}), (\textquotesingle{}elsqual\textquotesingle{}, \textquotesingle{}autre\textquotesingle{}), (\textquotesingle{}n\textquotesingle{}, \textquotesingle{}aucune\textquotesingle{}){]}\})\textgreater{}}}}
\end{fulllineitems}

\index{sau (attribut app.home.content\_gen.questionaire.QuestionairesAgri)@\spxentry{sau}\spxextra{attribut app.home.content\_gen.questionaire.QuestionairesAgri}}

\begin{fulllineitems}
\phantomsection\label{\detokenize{app.home.content_gen:app.home.content_gen.questionaire.QuestionairesAgri.sau}}\pysigline{\sphinxbfcode{\sphinxupquote{sau}}\sphinxbfcode{\sphinxupquote{ = \textless{}UnboundField(TextField, (\textquotesingle{}sau\textquotesingle{},), \{\})\textgreater{}}}}
\end{fulllineitems}

\index{submit (attribut app.home.content\_gen.questionaire.QuestionairesAgri)@\spxentry{submit}\spxextra{attribut app.home.content\_gen.questionaire.QuestionairesAgri}}

\begin{fulllineitems}
\phantomsection\label{\detokenize{app.home.content_gen:app.home.content_gen.questionaire.QuestionairesAgri.submit}}\pysigline{\sphinxbfcode{\sphinxupquote{submit}}\sphinxbfcode{\sphinxupquote{ = \textless{}UnboundField(SubmitField, (\textquotesingle{}Enregistrer\textquotesingle{},), \{\})\textgreater{}}}}
\end{fulllineitems}

\index{typecult (attribut app.home.content\_gen.questionaire.QuestionairesAgri)@\spxentry{typecult}\spxextra{attribut app.home.content\_gen.questionaire.QuestionairesAgri}}

\begin{fulllineitems}
\phantomsection\label{\detokenize{app.home.content_gen:app.home.content_gen.questionaire.QuestionairesAgri.typecult}}\pysigline{\sphinxbfcode{\sphinxupquote{typecult}}\sphinxbfcode{\sphinxupquote{ = \textless{}UnboundField(TextField, (\textquotesingle{}type culture\textquotesingle{},), \{\})\textgreater{}}}}
\end{fulllineitems}

\index{typefonc (attribut app.home.content\_gen.questionaire.QuestionairesAgri)@\spxentry{typefonc}\spxextra{attribut app.home.content\_gen.questionaire.QuestionairesAgri}}

\begin{fulllineitems}
\phantomsection\label{\detokenize{app.home.content_gen:app.home.content_gen.questionaire.QuestionairesAgri.typefonc}}\pysigline{\sphinxbfcode{\sphinxupquote{typefonc}}\sphinxbfcode{\sphinxupquote{ = \textless{}UnboundField(SelectField, (\textquotesingle{}type de foncier\textquotesingle{},), \{\textquotesingle{}choices\textquotesingle{}: {[}(\textquotesingle{}init\textquotesingle{}, \textquotesingle{}sélectionnez la proposition\textquotesingle{}), (\textquotesingle{}prop\textquotesingle{}, \textquotesingle{}proprietaire\textquotesingle{}), (\textquotesingle{}loc\textquotesingle{}, \textquotesingle{}locataire\textquotesingle{}), (\textquotesingle{}mist\textquotesingle{}, \textquotesingle{}proprietaire et locataire\textquotesingle{}){]}\})\textgreater{}}}}
\end{fulllineitems}


\end{fulllineitems}

\index{save\_data() (dans le module app.home.content\_gen.questionaire)@\spxentry{save\_data()}\spxextra{dans le module app.home.content\_gen.questionaire}}

\begin{fulllineitems}
\phantomsection\label{\detokenize{app.home.content_gen:app.home.content_gen.questionaire.save_data}}\pysiglinewithargsret{\sphinxcode{\sphinxupquote{app.home.content\_gen.questionaire.}}\sphinxbfcode{\sphinxupquote{save\_data}}}{\emph{\DUrole{n}{data}}}{}
\sphinxAtStartPar
Cette fonction enregistre les données du questionaire
\begin{quote}
\begin{quote}\begin{description}
\item[{return}] \leavevmode
\sphinxAtStartPar
dernières données rentrées pour l’affichage

\end{description}\end{quote}
\end{quote}
\begin{quote}\begin{description}
\item[{Type renvoyé}] \leavevmode
\sphinxAtStartPar
pandas df

\end{description}\end{quote}

\end{fulllineitems}



\subparagraph{Module contents}
\label{\detokenize{app.home.content_gen:module-app.home.content_gen}}\label{\detokenize{app.home.content_gen:module-contents}}\index{module@\spxentry{module}!app.home.content\_gen@\spxentry{app.home.content\_gen}}\index{app.home.content\_gen@\spxentry{app.home.content\_gen}!module@\spxentry{module}}

\paragraph{Submodules}
\label{\detokenize{app.home:submodules}}

\paragraph{app.home.routes module}
\label{\detokenize{app.home:module-app.home.routes}}\label{\detokenize{app.home:app-home-routes-module}}\index{module@\spxentry{module}!app.home.routes@\spxentry{app.home.routes}}\index{app.home.routes@\spxentry{app.home.routes}!module@\spxentry{module}}
\sphinxAtStartPar
Modified for GRID, 2021

\sphinxAtStartPar
Copyright (c) 2019 \sphinxhyphen{} present AppSeed.us
\index{get\_segment() (dans le module app.home.routes)@\spxentry{get\_segment()}\spxextra{dans le module app.home.routes}}

\begin{fulllineitems}
\phantomsection\label{\detokenize{app.home:app.home.routes.get_segment}}\pysiglinewithargsret{\sphinxcode{\sphinxupquote{app.home.routes.}}\sphinxbfcode{\sphinxupquote{get\_segment}}}{\emph{\DUrole{n}{request}}}{}
\end{fulllineitems}

\index{index() (dans le module app.home.routes)@\spxentry{index()}\spxextra{dans le module app.home.routes}}

\begin{fulllineitems}
\phantomsection\label{\detokenize{app.home:app.home.routes.index}}\pysiglinewithargsret{\sphinxcode{\sphinxupquote{app.home.routes.}}\sphinxbfcode{\sphinxupquote{index}}}{}{}
\end{fulllineitems}

\index{route\_template() (dans le module app.home.routes)@\spxentry{route\_template()}\spxextra{dans le module app.home.routes}}

\begin{fulllineitems}
\phantomsection\label{\detokenize{app.home:app.home.routes.route_template}}\pysiglinewithargsret{\sphinxcode{\sphinxupquote{app.home.routes.}}\sphinxbfcode{\sphinxupquote{route\_template}}}{\emph{\DUrole{n}{template}}}{}
\end{fulllineitems}



\paragraph{Module contents}
\label{\detokenize{app.home:module-app.home}}\label{\detokenize{app.home:module-contents}}\index{module@\spxentry{module}!app.home@\spxentry{app.home}}\index{app.home@\spxentry{app.home}!module@\spxentry{module}}
\sphinxAtStartPar
Modified for GRID, 2021

\sphinxAtStartPar
Copyright (c) 2019 \sphinxhyphen{} present AppSeed.us


\subsection{Module contents}
\label{\detokenize{app:module-0}}\label{\detokenize{app:module-contents}}\index{module@\spxentry{module}!app@\spxentry{app}}\index{app@\spxentry{app}!module@\spxentry{module}}
\sphinxAtStartPar
Modfied for GRID, 2021

\sphinxAtStartPar
Copyright (c) 2019 \sphinxhyphen{} present AppSeed.us
\index{configure\_database() (dans le module app)@\spxentry{configure\_database()}\spxextra{dans le module app}}

\begin{fulllineitems}
\phantomsection\label{\detokenize{app:app.configure_database}}\pysiglinewithargsret{\sphinxcode{\sphinxupquote{app.}}\sphinxbfcode{\sphinxupquote{configure\_database}}}{\emph{\DUrole{n}{app}}}{}
\end{fulllineitems}

\index{create\_app() (dans le module app)@\spxentry{create\_app()}\spxextra{dans le module app}}

\begin{fulllineitems}
\phantomsection\label{\detokenize{app:app.create_app}}\pysiglinewithargsret{\sphinxcode{\sphinxupquote{app.}}\sphinxbfcode{\sphinxupquote{create\_app}}}{\emph{\DUrole{n}{config}}}{}
\end{fulllineitems}

\index{register\_blueprints() (dans le module app)@\spxentry{register\_blueprints()}\spxextra{dans le module app}}

\begin{fulllineitems}
\phantomsection\label{\detokenize{app:app.register_blueprints}}\pysiglinewithargsret{\sphinxcode{\sphinxupquote{app.}}\sphinxbfcode{\sphinxupquote{register\_blueprints}}}{\emph{\DUrole{n}{app}}}{}
\end{fulllineitems}

\index{register\_extensions() (dans le module app)@\spxentry{register\_extensions()}\spxextra{dans le module app}}

\begin{fulllineitems}
\phantomsection\label{\detokenize{app:app.register_extensions}}\pysiglinewithargsret{\sphinxcode{\sphinxupquote{app.}}\sphinxbfcode{\sphinxupquote{register\_extensions}}}{\emph{\DUrole{n}{app}}}{}
\end{fulllineitems}



\chapter{Index et recherche}
\label{\detokenize{index:index-et-recherche}}\begin{itemize}
\item {} 
\sphinxAtStartPar
\DUrole{xref,std,std-ref}{genindex}

\item {} 
\sphinxAtStartPar
\DUrole{xref,std,std-ref}{modindex}

\item {} 
\sphinxAtStartPar
\DUrole{xref,std,std-ref}{search}

\end{itemize}


\renewcommand{\indexname}{Index des modules Python}
\begin{sphinxtheindex}
\let\bigletter\sphinxstyleindexlettergroup
\bigletter{a}
\item\relax\sphinxstyleindexentry{agri\_data}\sphinxstyleindexpageref{agri_data:\detokenize{module-agri_data}}
\item\relax\sphinxstyleindexentry{agri\_data.data\_draw}\sphinxstyleindexpageref{agri_data:\detokenize{module-agri_data.data_draw}}
\item\relax\sphinxstyleindexentry{agri\_data.data\_import}\sphinxstyleindexpageref{agri_data:\detokenize{module-agri_data.data_import}}
\item\relax\sphinxstyleindexentry{app}\sphinxstyleindexpageref{app:\detokenize{module-0}}
\item\relax\sphinxstyleindexentry{app.base}\sphinxstyleindexpageref{app.base:\detokenize{module-app.base}}
\item\relax\sphinxstyleindexentry{app.base.forms}\sphinxstyleindexpageref{app.base:\detokenize{module-app.base.forms}}
\item\relax\sphinxstyleindexentry{app.base.models}\sphinxstyleindexpageref{app.base:\detokenize{module-app.base.models}}
\item\relax\sphinxstyleindexentry{app.base.routes}\sphinxstyleindexpageref{app.base:\detokenize{module-app.base.routes}}
\item\relax\sphinxstyleindexentry{app.base.util}\sphinxstyleindexpageref{app.base:\detokenize{module-app.base.util}}
\item\relax\sphinxstyleindexentry{app.home}\sphinxstyleindexpageref{app.home:\detokenize{module-app.home}}
\item\relax\sphinxstyleindexentry{app.home.content\_gen}\sphinxstyleindexpageref{app.home.content_gen:\detokenize{module-app.home.content_gen}}
\item\relax\sphinxstyleindexentry{app.home.content\_gen.graph\_generation}\sphinxstyleindexpageref{app.home.content_gen:\detokenize{module-app.home.content_gen.graph_generation}}
\item\relax\sphinxstyleindexentry{app.home.content\_gen.index\_renderer}\sphinxstyleindexpageref{app.home.content_gen:\detokenize{module-app.home.content_gen.index_renderer}}
\item\relax\sphinxstyleindexentry{app.home.content\_gen.map\_generation}\sphinxstyleindexpageref{app.home.content_gen:\detokenize{module-app.home.content_gen.map_generation}}
\item\relax\sphinxstyleindexentry{app.home.content\_gen.questionaire}\sphinxstyleindexpageref{app.home.content_gen:\detokenize{module-app.home.content_gen.questionaire}}
\item\relax\sphinxstyleindexentry{app.home.routes}\sphinxstyleindexpageref{app.home:\detokenize{module-app.home.routes}}
\end{sphinxtheindex}

\renewcommand{\indexname}{Index}
\printindex
\end{document}